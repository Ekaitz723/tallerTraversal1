\markdownRendererDocumentBegin
\markdownRendererSectionBegin
\markdownRendererHeadingTwo{Informe de Análisis del Proyecto con Cppcheck}\markdownRendererInterblockSeparator
{}\markdownRendererSectionBegin
\markdownRendererHeadingThree{Clasificación de Hallazgos}\markdownRendererInterblockSeparator
{}\markdownRendererSectionBegin
\markdownRendererHeadingFour{Críticos:}\markdownRendererInterblockSeparator
{}\markdownRendererStrongEmphasis{doubleFree en hgibbs.cpp}: Se detectó un problema grave de liberación doble de memoria en las líneas 83 y 105, donde la variable \markdownRendererCodeSpan{v2} es liberada dos veces. Este tipo de error puede causar comportamiento indefinido y crashes en tiempo de ejecución.\markdownRendererInterblockSeparator
{}\markdownRendererUlBeginTight
\markdownRendererUlItem \markdownRendererStrongEmphasis{Acción tomada}: Se ha decidido ignorar temporalmente debido a que \markdownRendererCodeSpan{v2} se utiliza como "puntero recipiente" con llamadas a \markdownRendererCodeSpan{free()} controladas en los test.\markdownRendererUlItemEnd 
\markdownRendererUlItem \markdownRendererStrongEmphasis{Riesgo}: Alto - Puede causar corrupción de memoria y crashes.\markdownRendererUlItemEnd 
\markdownRendererUlEndTight \markdownRendererInterblockSeparator
{}
\markdownRendererSectionEnd \markdownRendererSectionBegin
\markdownRendererHeadingFour{De Estilo y Optimización}\markdownRendererInterblockSeparator
{}\markdownRendererSectionBegin
\markdownRendererHeadingFive{Problemas de Const}\markdownRendererInterblockSeparator
{}Se identificaron múltiples oportunidades para mejorar la inmutabilidad del código:\markdownRendererInterblockSeparator
{}\markdownRendererUlBeginTight
\markdownRendererUlItem \markdownRendererStrongEmphasis{constParameter} (Cheb3D.cpp:8): Parámetros de array que pueden declararse como const\markdownRendererUlItemEnd 
\markdownRendererUlItem \markdownRendererStrongEmphasis{constVariable} (EKF\markdownRendererUnderscore{}GEOS3.cpp:386): Array \markdownRendererCodeSpan{Y\markdownRendererUnderscore{}true} puede ser const \markdownRendererUlItemEnd 
\markdownRendererUlItem \markdownRendererStrongEmphasis{constVariablePointer} (EKF\markdownRendererEmphasis{GEOS3}test.cpp:271): Puntero \markdownRendererCodeSpan{results} puede ser const\markdownRendererUlItemEnd 
\markdownRendererUlItem ...\markdownRendererUlItemEnd 
\markdownRendererUlEndTight \markdownRendererInterblockSeparator
{}\markdownRendererStrongEmphasis{Estado}: Corregidos durante la revisión de código.\markdownRendererInterblockSeparator
{}
\markdownRendererSectionEnd \markdownRendererSectionBegin
\markdownRendererHeadingFive{Ignorados por igualdad con MATLAB}\markdownRendererInterblockSeparator
{}Para mantener la máxima similitud con la implementación original de MATLAB, se han ignorado intencionalmente los siguientes:\markdownRendererInterblockSeparator
{}knownConditionTrueFalse (DEInteg.cpp:84): Condición State\markdownRendererUnderscore{} > DE\markdownRendererEmphasis{INVPARAM siempre evalúa a false debido a la inicialización de State} como DE\markdownRendererEmphasis{INIT (valor 1). duplicateExpression (G}AccelHarmonic.cpp:23): Expresión 1.0 / d donde d = 1.0, resultando en una operación innecesaria. variableScope: Variables cuyo scope podría reducirse para mejorar la encapsulación shadowVariable: Variables que ocultan otras variables del mismo nombre en scopes superiores unreadVariable: Variables declaradas pero nunca leídas en el código duplicateAssignExpression: Expresiones de asignación duplicadas que podrían simplificarse\markdownRendererInterblockSeparator
{}\markdownRendererStrongEmphasis{Estado}: Tambien ignorados intencionalmente para mantener parecido con implementación original de MATLAB.\markdownRendererInterblockSeparator
{}
\markdownRendererSectionEnd \markdownRendererSectionBegin
\markdownRendererHeadingFive{Ignorados}\markdownRendererInterblockSeparator
{}
\markdownRendererSectionEnd \markdownRendererSectionBegin
\markdownRendererHeadingFive{Código Unused}\markdownRendererInterblockSeparator
{}\markdownRendererUlBeginTight
\markdownRendererUlItem \markdownRendererStrongEmphasis{unusedFunction} (anglesg.cpp:370): Función \markdownRendererCodeSpan{debug\markdownRendererUnderscore{}anglesg\markdownRendererUnderscore{}complete()} nunca utilizada.\markdownRendererUlItemEnd 
\markdownRendererUlEndTight \markdownRendererInterblockSeparator
{}\markdownRendererStrongEmphasis{Estado}: Mantenida para propósitos de debugging futuro.
\markdownRendererSectionEnd 
\markdownRendererSectionEnd 
\markdownRendererSectionEnd 
\markdownRendererSectionEnd \markdownRendererDocumentEnd