\markdownRendererDocumentBegin
El proyecto implementa un framework de testing robusto para validar la correcta migración desde MATLAB. Cada función traducida incluye tests unitarios exhaustivos que verifican múltiples escenarios:\markdownRendererInterblockSeparator
{}\markdownRendererStrongEmphasis{Tests unitarios}: Se declaran directamente en el mismo archivo .cpp de cada función, validando casos típicos (órbitas LEO, GEO, elípticas), casos límite (vectores colineales, magnitudes extremas) y verificación de propiedades físicas (conservación de energía, perpendicularidad en órbitas circulares). Se utilizan assertions con tolerancias numéricas apropiadas para comparar resultados contra valores de referencia conocidos. El framework utiliza macros \markdownRendererBackslash{}texttt\markdownRendererLeftBrace{}REGISTER\markdownRendererUnderscore{}TEST\markdownRendererRightBrace{} para ejecutar automáticamente todos los tests.\markdownRendererInterblockSeparator
{}\markdownRendererStrongEmphasis{Tests de integración}: Verifican el comportamiento de sistemas completos como \markdownRendererBackslash{}texttt\markdownRendererLeftBrace{}EKF\markdownRendererUnderscore{}GEOS3\markdownRendererUnderscore{}test\markdownRendererRightBrace{}, que valida la implementación completa del filtro de Kalman extendido aplicado al satelite GEOS3, integrando determinación orbital, propagación y estimación de estados. También incluyen funciones compuestas como \markdownRendererBackslash{}texttt\markdownRendererLeftBrace{}anglesg\markdownRendererRightBrace{}, que combina transformaciones de coordenadas, el método de Gibbs y cálculos iterativos. Estos tests detectan la propagación de errores a través de la cadena de dependencias y validan que los resultados finales coincidan con los obtenidos en MATLAB.\markdownRendererInterblockSeparator
{}La estrategia de testing fue fundamental para detectar errores sutiles causados por diferencias entre MATLAB y C++ en precisión numérica y manejo de casos especiales. \markdownRendererBackslash{}begin\markdownRendererLeftBrace{}markdown\markdownRendererRightBrace{}\markdownRendererInterblockSeparator
{}\markdownRendererBackslash{}section\markdownRendererLeftBrace{}Mejoras en el código\markdownRendererRightBrace{}\markdownRendererInterblockSeparator
{}Para garantizar la calidad y rendimiento de la librería de astrodynamica, se implementó un proceso sistemático de análisis y optimización del código mediante herramientas especializadas de profiling y análisis estático.\markdownRendererInterblockSeparator
{}\markdownRendererBackslash{}subsection\markdownRendererLeftBrace{}Metodología de análisis\markdownRendererRightBrace{}\markdownRendererInterblockSeparator
{}El proceso de mejora se ejecutó en tres fases diferenciadas:\markdownRendererInterblockSeparator
{}\markdownRendererBackslash{}begin\markdownRendererLeftBrace{}enumerate\markdownRendererRightBrace{} \markdownRendererBackslash{}item \markdownRendererBackslash{}textbf\markdownRendererLeftBrace{}Análisis estático pre-ejecución\markdownRendererRightBrace{}: Utilizando CppCheck para identificar problemas potenciales en el código fuente sin necesidad de compilación o ejecución.\markdownRendererInterblockSeparator
{}\markdownRendererBackslash{}item \markdownRendererBackslash{}textbf\markdownRendererLeftBrace{}Análisis de rendimiento\markdownRendererRightBrace{}: Mediante Gprof para identificar cuellos de botella y patrones de uso de funciones durante la ejecución real del programa.\markdownRendererInterblockSeparator
{}\markdownRendererBackslash{}item \markdownRendererBackslash{}textbf\markdownRendererLeftBrace{}Análisis de memoria\markdownRendererRightBrace{}: Con Valgrind para detectar memory leaks, buffer overflows y otros problemas relacionados con la gestión dinámica de memoria. \markdownRendererBackslash{}end\markdownRendererLeftBrace{}enumerate\markdownRendererRightBrace{}\markdownRendererInterblockSeparator
{}Cada herramienta proporcionó perspectivas complementarias sobre diferentes aspectos del código, permitiendo una evaluación integral de la calidad del software desarrollado.\markdownRendererInterblockSeparator
{}\markdownRendererBackslash{}subsection\markdownRendererLeftBrace{}Resultados de los análisis\markdownRendererRightBrace{}\markdownRendererInterblockSeparator
{}Los informes detallados de cada herramienta se presentan a continuación, incluyendo las métricas específicas, problemas identificados y mejoras implementadas.\markdownRendererInterblockSeparator
{}\markdownRendererPercentSign{} Informes de análisis de código \markdownRendererBackslash{}markdownSetup\markdownRendererLeftBrace{}shiftHeadings = 2\markdownRendererRightBrace{} \markdownRendererBackslash{}markdownInput\markdownRendererLeftBrace{}informeCppcheck.md\markdownRendererRightBrace{}\markdownRendererInterblockSeparator
{}\markdownRendererBackslash{}markdownSetup\markdownRendererLeftBrace{}shiftHeadings = 2\markdownRendererRightBrace{} \markdownRendererBackslash{}markdownInput\markdownRendererLeftBrace{}informeGprof.md\markdownRendererRightBrace{}\markdownRendererInterblockSeparator
{}\markdownRendererBackslash{}markdownSetup\markdownRendererLeftBrace{}shiftHeadings = 2\markdownRendererRightBrace{} \markdownRendererBackslash{}markdownInput\markdownRendererLeftBrace{}informeValgrind.md\markdownRendererRightBrace{}\markdownRendererInterblockSeparator
{}\markdownRendererBackslash{}subsection\markdownRendererLeftBrace{}Impacto de las mejoras\markdownRendererRightBrace{}\markdownRendererInterblockSeparator
{}La aplicación sistemática de estos análisis permitió identificar y resolver issues específicos, resultando en un código más robusto y eficiente que mantiene la precisión numérica requerida. \markdownRendererBackslash{} Aún así, fui incapaz de conseguir el valor exacto, habiendo un pequeñisimo error de redondeo que no logro encontrar desde donde ocurre.\markdownRendererInterblockSeparator
{}\markdownRendererBackslash{}section\markdownRendererLeftBrace{}Documentacion\markdownRendererRightBrace{} Para la generación de documentación técnica se utilizó Doxygen. La configuración clave del Doxyfile incluyó: \markdownRendererBackslash{}begin\markdownRendererLeftBrace{}center\markdownRendererRightBrace{} \markdownRendererBackslash{}begin\markdownRendererLeftBrace{}markdown\markdownRendererRightBrace{} \markdownRendererCodeSpan{
EXTRACT\markdownRendererUnderscore{}ALL = YES
EXTRACT\markdownRendererUnderscore{}PRIVATE = YES
EXTRACT\markdownRendererUnderscore{}STATIC = YES
SOURCE\markdownRendererUnderscore{}BROWSER = YES
INLINE\markdownRendererUnderscore{}SOURCES = YES
CALL\markdownRendererUnderscore{}GRAPH = YES
CALLER\markdownRendererUnderscore{}GRAPH = YES
CLASS\markdownRendererUnderscore{}DIAGRAMS = YES
COLLABORATION\markdownRendererUnderscore{}GRAPH = YES
}\markdownRendererDocumentEnd