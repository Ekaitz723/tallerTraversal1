\markdownRendererDocumentBegin
\markdownRendererSectionBegin
\markdownRendererHeadingOne{Problemas durante el proyecto}\markdownRendererInterblockSeparator
{}Estos fueron los problemas con más ocurrencias y/o más dificultosos durante la realización del proyecto:\markdownRendererInterblockSeparator
{}\markdownRendererSectionBegin
\markdownRendererHeadingTwo{\markdownRendererStrongEmphasis{Indexación base-1 vs base-0}:}\markdownRendererInterblockSeparator
{}La diferencia de indexación entre MATLAB (base 1) y C++ (base 0) causó errores sutiles pero críticos cuando me despistaba. La clase Matrix funciona en base 1 de indexación, por lo que esos errores me los ahorré, pero en el resto del código tuve que tener cuidado de ello.\markdownRendererInterblockSeparator
{}
\markdownRendererSectionEnd \markdownRendererSectionBegin
\markdownRendererHeadingTwo{\markdownRendererStrongEmphasis{Orientación de vectores}:}\markdownRendererInterblockSeparator
{}MATLAB maneja vectores fila y columna de forma implícita, pero en C++ hay que ser explícito. Los problemas de concatenación vertical vs horizontal no fueron muy dificultosos, pero molestaron bastante.\markdownRendererInterblockSeparator
{}
\markdownRendererSectionEnd \markdownRendererSectionBegin
\markdownRendererHeadingTwo{\markdownRendererStrongEmphasis{Propagación silenciosa de errores}:}\markdownRendererInterblockSeparator
{}El problema más insidioso fue que al tener funciones que dependen unas de otras (como \markdownRendererCodeSpan{anglesg} que llama a \markdownRendererCodeSpan{gibbs}, que usa \markdownRendererCodeSpan{angl} y \markdownRendererCodeSpan{unit}), un pequeño error en una función base se propagaba en cascada a través de toda la cadena de cálculos. Los errores se acumulaban y se manifestaban como resultados aparentemente correctos pero numericamente incorrectos varios niveles arriba, haciendo el debugging extremadamente complejo porque el fallo real estaba enterrado en funciones de bajo nivel.\markdownRendererInterblockSeparator
{}
\markdownRendererSectionEnd \markdownRendererSectionBegin
\markdownRendererHeadingTwo{\markdownRendererStrongEmphasis{Precisión numérica}:}\markdownRendererInterblockSeparator
{}Diferencias mínimas en cálculos de punto flotante entre MATLAB y C++ causaron divergencias en algoritmos iterativos. Se ve en el código que tuvieron que ajustar tolerancias y criterios de convergencia.\markdownRendererInterblockSeparator
{}
\markdownRendererSectionEnd \markdownRendererSectionBegin
\markdownRendererHeadingTwo{\markdownRendererStrongEmphasis{Condiciones de error}:}\markdownRendererInterblockSeparator
{}Las cadenas de error exactas (' ok' vs 'not coplanar') tenían que coincidir al caracter, incluyendo espacios. Los algoritmos como \markdownRendererCodeSpan{gibbs} dependían de estas comparaciones exactas.\markdownRendererInterblockSeparator
{}
\markdownRendererSectionEnd \markdownRendererSectionBegin
\markdownRendererHeadingTwo{\markdownRendererStrongEmphasis{Testing y validación}:}\markdownRendererInterblockSeparator
{}Verificar que cada función traduzida produjera exactamente los mismos resultados que MATLAB requirió crear casos de prueba exhaustivos y frameworks de comparación robustos.
\markdownRendererSectionEnd 
\markdownRendererSectionEnd \markdownRendererDocumentEnd