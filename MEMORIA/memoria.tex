\documentclass[12pt,a4paper]{article}

% Configuración basica
\usepackage[utf8]{inputenc}
\usepackage[T1]{fontenc}
\usepackage[spanish]{babel}
\usepackage[margin=2.5cm]{geometry}

% Librería de markdown (version minima)
\usepackage{markdown}

% Paquetes adicionales 
\usepackage{xcolor}
\usepackage{graphicx}
\usepackage{float}
\usepackage{hyperref}
\usepackage{listings}
\usepackage{fancyhdr}
\usepackage{titlesec}
\usepackage{enumitem}
\usepackage{setspace}
\usepackage{tcolorbox}
\onehalfspacing

% Configuración de colores
\definecolor{codeblue}{RGB}{42, 161, 152}
\definecolor{codegray}{RGB}{128, 128, 128}
\definecolor{lightgray}{RGB}{245, 245, 245}

% Configuración de markdown
\markdownSetup{
    shiftHeadings = 1,
    rendererPrototypes = {
        link = {\href{#2}{#1}},
        emphasis = {\emph{#1}},
        strongEmphasis = {\textbf{#1}},
        % image = {\begin{center}\includegraphics[width=0.8\textwidth]{#2}\end{center}},
        image = {\begin{figure}[H]\centering\includegraphics[width=0.8\textwidth]{#2}\caption{#1}\end{figure}},
        codeSpan = {\texttt{#1}},
    },
    fencedCode = true,
    fencedCodeAttributes = true,
    blankBeforeCodeFence = true,
    pipeTables = true,
    tableCaptions = true,
}


% Configuración para código
\lstset{
    backgroundcolor=\color{lightgray},
    basicstyle=\ttfamily\small,
    breakatwhitespace=false,
    breaklines=true,
    captionpos=b,
    commentstyle=\color{codegray},
    frame=single,
    keepspaces=true,
    keywordstyle=\color{codeblue},
    numbers=left,
    numbersep=5pt,
    numberstyle=\tiny\color{codegray},
    rulecolor=\color{black},
    showspaces=false,
    showstringspaces=false,
    showtabs=false,
    stepnumber=1,
    stringstyle=\color{red},
    tabsize=2
}

% Configuración de cajas para destacar contenido
\newtcolorbox{infobox}[1][Información]{
    colback=blue!5!white,
    colframe=blue!75!black,
    title=#1,
    fonttitle=\bfseries
}

\newtcolorbox{warningbox}[1][Advertencia]{
    colback=orange!5!white,
    colframe=orange!75!black,
    title=#1,
    fonttitle=\bfseries
}

% Configuración de encabezados y pies de página
\pagestyle{fancy}
\fancyhf{}
\fancyhead[L]{\leftmark}
\fancyhead[R]{\thepage}
\fancyfoot[C]{Memoria del Proyecto - Taller Transversal I}

% Configuración de títulos de sección
\titleformat{\section}
{\normalfont\Large\bfseries\color{blue!70!black}}
{\thesection}{1em}{}

\titleformat{\subsection}
{\normalfont\large\bfseries\color{blue!50!black}}
{\thesubsection}{1em}{}
\hypersetup{
    colorlinks=true,
    linkcolor=blue,
    filecolor=magenta,
    urlcolor=cyan,
    citecolor=red
}

% Comando personalizado para separar secciones
\newcommand{\separadorseccion}{%
    \vspace{1em}
    \hrule
    \vspace{1em}
}

\title{Memoria del Proyecto de Taller Transversal I}
\author{Ekaitz Arriola Garcia}
\date{\today}

\begin{document}

\maketitle
\tableofcontents
\newpage




% INTRODUCCION
\section{Introducción}

Este proyecto aborda la migración sistemática de una librería de astrodynamica desde MATLAB hacia C++, preservando la precisión numerica y funcionalidad del código original. La librería implementa algoritmos fundamentales para determinación orbital, propagación de trayectorias satelitales y conversiones entre sistemas de coordenadas, utilizando una clase Matrix personalizada que replica el comportamiento matricial nativo de MATLAB. 
\\El desarrollo incluye la traducción de funciones críticas como Gibbs para determinación orbital inicial, cálculo de elementos orbitales keplerianos, y transformaciones geodésicas, manteniendo compatibilidad directa con los algoritmos originales mediante indexación base-1 y operaciones vectoriales equivalentes. La implementación garantiza la integridad de los resultados através de un framework de testing exhaustivo que valida cada función traducida contra casos conocidos y ejemplos de referencia en mecánica orbital.
\\\\
Todos los recursos nombrados en la memoria se podrá acceder desde \href{https://github.com/Ekaitz723/tallerTraversal1/}{Github}.


\section{Tiempos del proyecto}
% % Tiempo estimado {
    % \begin{center}
    %     \begin{tabular}{|l|c|c|}
    %         \hline
    %         \textbf{Tarea} & \textbf{Tiempo estimado} & \textbf{Tiempo real} \\
    %         \hline
    %          &  minutos &  minutos \\
    %         \hline
    %     \end{tabular}
    % \end{center}
    % \begin{center}
    %     Comentario sobre la tarea
    % \end{center}
% % }
% Uno de esos por cada codigo. Esto al final, primero hacé los otros cosos
% Recuerda que << no lo hiciste

\begin{center}
    \begin{tabular}{|p{3cm}|c|c|p{6cm}|}
        \hline
        \textbf{Tarea} & \textbf{Tiempo estimado} & \textbf{Tiempo real} & \textbf{Comentario} \\
        \hline
        \textbf{Matrix} & \textbf{190 min} & \textbf{145 min} & \\
        \hline
        zeros(row, column) & 5 min & 5 min & Tiempo estimado basado en experiencia previa en tareas similares.\\
        \hline
        operator \- (Matrix) & 5 min & 5 min & Tiempo estimado basado en experiencia previa en tareas similares.\\
        \hline
        operator \<\< (Matrix) & 0 min & 0 min & No completado. \\
        \hline
        operator / (Matrix) & 5 min & 5 min & Tiempo estimado basado en experiencia previa en tareas similares.\\
        \hline
        operator () (row, column) & 5 min & 5 min & Tiempo estimado basado en experiencia previa en tareas similares.\\
        \hline
        operator + (Matrix) & 5 min & 5 min & Tiempo estimado basado en experiencia previa en tareas similares.\\
        \hline
        inv(Matrix) & 5 min & 5 min & Tiempo estimado basado en experiencia previa en tareas similares.\\
        \hline
        transpose(Matrix) & 5 min & 5 min & Tiempo estimado basado en experiencia previa en tareas similares.\\
        \hline
        eye(n) & 5 min & 5 min & Tiempo estimado basado en experiencia previa en tareas similares.\\
        \hline
        operator = (Matrix) & 10 min & 5 min & No me acordaba de como definir este operador.\\
        \hline
        Matrix(row, column) & 5 min & 5 min & Tiempo estimado basado en experiencia previa en tareas similares.\\
        \hline
        operator * (Matrix) & 5 min & 5 min & Tiempo estimado basado en experiencia previa en tareas similares.\\
        \hline
        operator + (Matrix) & 5 min & 5 min & Tiempo estimado basado en experiencia previa en tareas similares.\\
        \hline
        operator - (Matrix) & 5 min & 5 min & Tiempo estimado basado en experiencia previa en tareas similares.\\
        \hline
        operator * (double) & 5 min & 5 min & Tiempo estimado basado en experiencia previa en tareas similares.\\
        \hline
        operator + (double) & 5 min & 5 min & Tiempo estimado basado en experiencia previa en tareas similares.\\
        \hline
        operator - (double) & 5 min & 5 min & Tiempo estimado basado en experiencia previa en tareas similares.\\
        \hline
        Matrix(n) & 5 min & 5 min & Tiempo estimado basado en experiencia previa en tareas similares.\\
        \hline
    \end{tabular}
 \end{center}


 \begin{center}
    \begin{tabular}{|p{3cm}|c|c|p{6cm}|}
        \hline
        \textbf{Tarea} & \textbf{Tiempo estimado} & \textbf{Tiempo real} & \textbf{Comentario} \\
        \hline
        operator () (n) & 5 min & 5 min & Tiempo estimado basado en experiencia previa en tareas similares.\\
        \hline
        zeros(n) & 5 min & 5 min & Tiempo estimado basado en experiencia previa en tareas similares.\\
        \hline
        norm & 10 min & 5 min & Tiempo estimado basado en experiencia previa en tareas similares.\\
        \hline
        dot & 10 min & 5 min & Tiempo estimado basado en experiencia previa en tareas similares.\\
        \hline
        cross & 10 min & 5 min & Tiempo estimado basado en experiencia previa en tareas similares.\\
        \hline
        extract\_vector & 10 min & 5 min & Tiempo estimado basado en experiencia previa en tareas similares.\\
        \hline
        union\_vector & 10 min & 10 min & Gestionar vecticales y horizontales.\\
        \hline
        extract\_row & 10 min & 5 min & Tiempo estimado basado en experiencia previa en tareas similares.\\
        \hline
        extract\_column & 10 min & 5 min & Tiempo estimado basado en experiencia previa en tareas similares.\\
        \hline
        assign\_row & 10 min & 5 min & Tiempo estimado basado en experiencia previa en tareas similares.\\
        \hline
        assign\_column & 10 min & 5 min & Tiempo estimado basado en experiencia previa en tareas similares.\\
        \hline
        \textbf{Iteration 1} & \textbf{240 min} & \textbf{350 min} & \\
        \hline
        AccelPointMass & 15 min & 20 min & Tiempo estimado basado en expreriencia previa, junto a realización de los test.\\
        \hline
        Cheb3D & 15 min & 20 min & Tiempo estimado basado en expreriencia previa, junto a realización de los test.\\
        \hline
        EccAnom & 15 min & 20 min & Tiempo estimado basado en expreriencia previa, junto a realización de los test.\\
        \hline
        Frac & 15 min & 20 min & Tiempo estimado basado en expreriencia previa, junto a realización de los test.\\
        \hline
    \end{tabular}
 \end{center}


 \begin{center}
    \begin{tabular}{|p{3cm}|c|c|p{6cm}|}
        \hline
        \textbf{Tarea} & \textbf{Tiempo estimado} & \textbf{Tiempo real} & \textbf{Comentario} \\
        \hline
        MeanObliquity & 15 min & 20 min & Tiempo estimado basado en expreriencia previa, junto a realización de los test.\\
        \hline
        Mjday & 15 min & 20 min & Tiempo estimado basado en expreriencia previa, junto a realización de los test.\\
        \hline
        Mjday\_TBD & 15 min & 20 min & Tiempo estimado basado en expreriencia previa, junto a realización de los test.\\
        \hline
        Position & 15 min & 20 min & Tiempo estimado basado en expreriencia previa, junto a realización de los test.\\
        \hline
        R\_x & 5 min & 10 min & Tiempo estimado basado en expreriencia previa, junto a realización de los test.\\
        \hline
        R\_y & 5 min & 10 min & Tiempo estimado basado en expreriencia previa, junto a realización de los test.\\
        \hline
        R\_z & 5 min & 10 min & Tiempo estimado basado en expreriencia previa, junto a realización de los test.\\
        \hline
        SAT\_Const & 0 min & 0 min & Ya estaba hecho.\\
        \hline
        sign\_ & 15 min & 20 min & Tiempo estimado basado en expreriencia previa, junto a realización de los test.\\
        \hline
        timediff & 15 min & 20 min & Tiempo estimado basado en expreriencia previa, junto a realización de los test.\\
        \hline
        AzElPa & 15 min & 20 min & Tiempo estimado basado en expreriencia previa, junto a realización de los test.\\
        \hline
        IERS & 15 min & 20 min & Tiempo estimado basado en expreriencia previa, junto a realización de los test.\\
        \hline
    \end{tabular}
 \end{center}



 \begin{center}
    \begin{tabular}{|p{3cm}|c|c|p{6cm}|}
        \hline
        \textbf{Tarea} & \textbf{Tiempo estimado} & \textbf{Tiempo real} & \textbf{Comentario} \\
        \hline
        Legendre & 15 min & 20 min & Tiempo estimado basado en expreriencia previa, junto a realización de los test.\\
        \hline
        NutAngles & 15 min & 20 min & Tiempo estimado basado en expreriencia previa, junto a realización de los test.\\
        \hline
        TimeUpdate & 15 min & 20 min & Tiempo estimado basado en expreriencia previa, junto a realización de los test.\\
        \hline
        \textbf{Iteration 2} & \textbf{160 min} & \textbf{160 min} & \\
        \hline
        AccelHarmonic & 20 min & 20 min & Tiempo estimado basado en expreriencia previa, junto a realización de los test.\\
        \hline
        EqnEquinox & 20 min & 20 min & Tiempo estimado basado en expreriencia previa, junto a realización de los test.\\
        \hline
        JPL\_Eph\_DE430 & 20 min & 20 min & Tiempo estimado basado en expreriencia previa, junto a realización de los test.\\
        \hline
        LTC & 20 min & 20 min & Tiempo estimado basado en expreriencia previa, junto a realización de los test.\\
        \hline
        NutMatrix & 20 min & 20 min & Tiempo estimado basado en expreriencia previa, junto a realización de los test.\\
        \hline
        PoleMatrix & 20 min & 20 min & Tiempo estimado basado en expreriencia previa, junto a realización de los test.\\
        \hline
        PrecMatrix & 20 min & 20 min & Tiempo estimado basado en expreriencia previa, junto a realización de los test.\\
        \hline
        gmst & 20 min & 20 min & Tiempo estimado basado en expreriencia previa, junto a realización de los test.\\
        \hline
    \end{tabular}
\end{center}

\begin{center}
    \begin{tabular}{|p{3cm}|c|c|p{6cm}|}
        \hline
        \textbf{Tarea} & \textbf{Tiempo estimado} & \textbf{Tiempo real} & \textbf{Comentario} \\
        \hline
        \textbf{Iteration 3} & \textbf{120 min} & \textbf{220 min} & \\
        \hline
        gast & 20 min & 20 min & Tiempo estimado basado en expreriencia previa, junto a realización de los test.\\
        \hline
        MeasUpdate & 20 min & 20 min & Tiempo estimado basado en expreriencia previa, junto a realización de los test.\\
        \hline
        G\_AccelHarmonic & 20 min & 40 min & Tiempo estimado basado en expreriencia previa, junto a realización de los test.\\
        \hline
        GHAMatrix & 20 min & 20 min & Tiempo estimado basado en expreriencia previa, junto a realización de los test.\\
        \hline
        Accel & 20 min & 60 min & Tiempo estimado basado en expreriencia previa, junto a realización de los test.\\
        \hline
        VarEqn & 20 min & 60 min & Tiempo estimado basado en expreriencia previa, junto a realización de los test.\\
        \hline
        \textbf{Iteration 4} & \textbf{90 min} & \textbf{140 min} & \\
        \hline
        DEInteg & 40 min & 60 min & Tiempo estimado basado en expreriencia previa, junto a realización de los test.\\
        \hline
        EKF\_GEOS3 & 50 min & 80 min & Tiempo estimado basado en expreriencia previa, junto a realización de los test.\\
        \hline
        
    \end{tabular}
\end{center}



\begin{center}
    \begin{tabular}{|p{3cm}|c|c|p{6cm}|}
        \hline
        \textbf{Tarea} & \textbf{Tiempo estimado} & \textbf{Tiempo real} & \textbf{Comentario} \\
        \hline
        \textbf{Extra} & \textbf{140 min} & \textbf{210 min} & \\
        \hline
        Geodetic & 20 min & 20 min & Tiempo estimado basado en expreriencia previa, junto a realización de los test.\\
        \hline
        angl & 20 min & 20 min & Tiempo estimado basado en expreriencia previa, junto a realización de los test.\\
        \hline
        unit & 20 min & 20 min & Tiempo estimado basado en expreriencia previa, junto a realización de los test.\\
        \hline
        anglesg & 20 min & 50 min & Tiempo estimado basado en expreriencia previa, junto a realización de los test.\\
        \hline
        elements & 20 min & 40 min & Tiempo estimado basado en expreriencia previa, junto a realización de los test.\\
        \hline
        gibbs & 20 min & 30 min & Tiempo estimado basado en expreriencia previa, junto a realización de los test.\\
        \hline
        hgibbs & 20 min & 30 min & Tiempo estimado basado en expreriencia previa, junto a realización de los test.\\
        \hline
        \textbf{Total} & \textbf{960 min} & \textbf{1245 min} & \\
        \hline
    \end{tabular}
\end{center}

\subsection{Horas totales reales}
20 horas y 45 minutos en total.

%  Clase Matrix base &  min &  min & Tiempo estimado basado en experiencia previa en tareas similares. \\

%  
    % \textbf{Iteration 3} & \textbf{ min} & \textbf{ min} & \\
    % \hline
%  \textbf{Iteration 4} & \textbf{ min} & \textbf{ min} & \\
%  \hline
%  
%  \textbf{Total} & \textbf{ min} & \textbf{ min} & \\






\section{Tiempos de ejecución}

\begin{table}[H]
\centering
\begin{tabular}{|l|c|c|}
\hline
Implementación & Matlab (original) & C++ \\
\hline
\textbf{Tiempo (segundos)} & 511.273 & 1.142 \\
\hline
\end{tabular}
\caption{Comparativa de tiempos de ejecución EKF\_GEOS3}
\label{tab:tiempos}
\end{table}
Matlab es muy lento. La implementación en cpp funciona casi 500 veces más veloz.
\\Esto se debe a que Matlab es un lenguaje interpretado que ejecuta código a través de su máquina virtual, mientras que C++ se compila directamente a código máquina.

\markdownSetup{shiftHeadings = 0}
\begin{markdown}
# Problemas y dificultades durante el proyecto

Estos fueron los problemas con más ocurrencias y más mayores dificultades durante la realización del proyecto:


## **Indexación base-1 vs base-0**: 

La diferencia de indexación entre MATLAB (base 1) y C++ (base 0) causó errores sutiles pero críticos cuando me despistaba. La clase Matrix funciona en base 1 de indexación, por lo que esos errores me los ahorré, pero en el resto del código tuve que tener cuidado de ello.


## **Orientación de vectores**: 

MATLAB maneja vectores fila y columna de forma implícita, pero en C++ hay que ser explícito. Los problemas de concatenación vertical vs horizontal no fueron muy dificultosos, pero molestaron bastante.


## **Propagación silenciosa de errores**: 

El mayor problema fue que al tener funciones que dependen unas de otras (como `anglesg` que llama a `gibbs`, que usa `angl` y `unit`), un pequeño error en una función base se propagaba en cascada a través de toda la cadena de cálculos. Los errores se acumulaban y se manifestaban como resultados aparentemente correctos pero numericamente incorrectos varios niveles arriba, haciendo el debugging extremadamente complejo porque el fallo real estaba enterrado en funciones de bajo nivel. A nivel test unitario, quizas pasaba con mucha permisividad, y el error se va acumulando.


## **Precisión numérica**: 

Diferencias mínimas en cálculos de punto flotante entre MATLAB y C++ causaron divergencias en algoritmos iterativos.


## **Condiciones de error**: 

Las cadenas de error exactas ('          ok' vs 'not coplanar') tenían que coincidir al caracter, incluyendo espacios. Los algoritmos como `gibbs` dependían de estas comparaciones exactas.


## **Testing y validación**: 

Verificar que cada función traducida produjera exactamente los mismos resultados que MATLAB requirió crear casos de prueba exhaustivos y frameworks de comparación robustos.
\end{markdown}


\section{Test unitarios y de integración}
% 
\markdownSetup{shiftHeadings = 0}
\markdownInput{test.md}


\section{Mejoras en el código}

Para garantizar la calidad y rendimiento se implementó un proceso sistemático de análisis y optimización del código mediante herramientas especializadas de profiling y análisis estático.


\subsection{Metodología de análisis}

El proceso de mejora se ejecutó en tres fases diferenciadas:

\begin{enumerate}
\item \textbf{Análisis estático pre-ejecución}: Utilizando CppCheck para identificar problemas potenciales en el código fuente sin necesidad de compilación o ejecución.

\item \textbf{Análisis de rendimiento}: Mediante Gprof para identificar cuellos de botella y patrones de uso de funciones durante la ejecución real del programa.

\item \textbf{Análisis de memoria}: Con Valgrind para detectar memory leaks, buffer overflows y otros problemas relacionados con la gestión dinámica de memoria.
\end{enumerate}

Cada herramienta proporcionó perspectivas complementarias sobre diferentes aspectos del código, permitiendo una evaluación integral de la calidad del software desarrollado.

\subsection{Resultados de los análisis}

Los informes detallados de cada herramienta se presentan a continuación, incluyendo las métricas específicas, problemas identificados y mejoras implementadas.

% Informes de análisis de código
\markdownSetup{shiftHeadings = 2}
\markdownInput{informeCppcheck.md}

\markdownSetup{shiftHeadings = 2}
\markdownInput{informeGprof.md}

\markdownSetup{shiftHeadings = 2}
\markdownInput{informeValgrind.md}

\subsection{Impacto de las mejoras}

La aplicación sistemática de estos análisis permitió identificar y resolver issues específicos, resultando en un código más robusto y eficiente que mantiene la precisión numérica requerida.
\\
Aún así, fui incapaz de conseguir el valor exacto, habiendo un pequeñisimo error de redondeo que no logro encontrar desde donde ocurre.

\section{Documentacion}
Para la generación de documentación técnica se utilizó Doxygen.
La configuración clave del Doxyfile incluyó:

\markdownSetup{shiftHeadings = 0}
\markdownInput{configuracionCppcheck.md}

Estas configuraciones permiten que Doxygen extraiga automáticamente la estructura completa de clases, funciones, parámetros y dependencias directamente del código. Se generaron diagramas de llamadas entre funciones y grafos de colaboración que resultan especialmente útiles para entender las interdependencias del proyecto.
\\Adicionalmente, se creó un archivo mainpage.md que proporciona una descripción general del proyecto, incluyendo funcionalidades principales, arquitectura del sistema, instrucciones de compilación y información del entorno de desarrollo. Este mainpage sirve como punto de entrada a la documentación generada.
\\La documentación resultante incluye navegación por código fuente, índices alfabéticos de funciones y una vista completa de la arquitectura del proyecto.

\end{document}